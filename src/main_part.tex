\stepcounter{mysection}\section{\arabic{mysection} Программная реализация}

	В качестве реализации языка Prolog был выбран SWI-Prolog.
		Это открытая реализация, работающая на Unix, MacOS и Windows.
		В данной реализации имеется необходимы функционал, а именно предикаты nb\_getval, nb\_setval и nb\_current.
		Эти предикаты позволяют использовать другие предикаты в качестве глобальных переменных, что
		позволяет существенно упростить написание правил для вычисления.

	В процессе реализации было создано порядка десяти файлов с описанием правил логического вывода
		общей суммой порядка семиста строк. Такое большое количество правил обусловлено наличием деталей и нюансов.
		Ниже представлен основной файл, в котором реализован инициализация логического вывода.
		Данный код иллюстрирует реализацию формулы времени. Благодаря функционалу SWI-Prolog
		в дальнейшем предикаты отражающие время можно будет указывать в правилах только при необходимости.

\input{src/pl_mod_code.tex}

	Реализации правил manage\_people и manage\_elevators вынесены в другие файлы,
		как как являются весьма громоздкими.
		Manage\_people отвечает за внешний фактор (случайное появление нуждающихся в лифте людей).
		А manage\_elevators включает в себя формулы движения лифтов.
