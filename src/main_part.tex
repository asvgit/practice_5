\stepcounter{mysection}\section{\arabic{mysection} Программная реализация}

	В качестве реализации языка Prolog был выбран SWI-Prolog.
		Это открытая реализация, работающая на Unix, MacOS и Windows.
		В данной реализации имеется необходимы функционал, а именно предикаты nb\_getval, nb\_setval и nb\_current.
		Эти предикаты позволяют использовать другие предикаты в качестве глобальных переменных, что
		позволяет существенно упростить написание правил для вычисления.

	В процессе реализации было создано порядка десяти файлов с описанием правил логического вывода
		общей суммой порядка семиста строк. Такое большое количество правил обусловлено наличием деталей и нюансов.
		Ниже представлен основной файл, в котором реализован инициализация логического вывода.
		Данный код иллюстрирует реализацию формулы времени. Благодаря функционалу SWI-Prolog
		в дальнейшем предикаты отражающие время можно будет указывать в правилах только при необходимости.


\begin{lstlisting}[basicstyle=\scriptsize]
:- ensure_loaded('conf.pl').
:- ensure_loaded('log.pl').
:- ensure_loaded('util.pl').
:- ensure_loaded('peopletool.pl').
:- ensure_loaded('elevatortool.pl').
:- ensure_loaded('maintool.pl').
:- ensure_loaded('init.pl').
do_loop_step :-
	manage_people,
	manage_elevators.
do_loop :-
	(nb_current(step, _) ->
		false
	;
		logwarn('Var \'Step\' has null value'),
		loginfo('Set first step \'0\''),
		nb_setval(step, 0)
	),
	do_loop.
do_loop :-
	var_getvalue(step, Step),
	nb_getval(n_steps, Steps),
	Step >= Steps.
do_loop :-
	var_getvalue(step, Step),
	nb_getval(n_steps, Steps),
	Step < Steps,
	logtrace('Start step'),
	do_loop_step,
	logtrace('Finish step'),
	Next is Step + 1,
	var_setvalue(step, Next),
	do_loop.
run:-
	show_var,
	logtrace('Start modulation'),
	init,
	do_loop,
	logtrace('Finith modulation'),
	show_stat.
% GOAL
:- run.
\end{lstlisting}


	Реализации правил manage\_people и manage\_elevators вынесены в другие файлы,
		как как являются весьма громоздкими.
		Manage\_people отвечает за внешний фактор (случайное появление нуждающихся в лифте людей).
		А manage\_elevators включает в себя формулы движения лифтов.
