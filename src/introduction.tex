\section{Задание на практику}
	Разработать модуляцию работы системы (группы) лифтов, используя тривиальный алгоритм обработки запросов.

	А также в ходе практики должны быть освоены компетенции:
		\begin{itemize}
			\item использованием на практике умений и навыков в организации исследовательских и проектных работ, в управлении коллективом;
			\item способность к самостоятельному обучению новым методам исследования, к изменению научного и научно-производственного профиля своей профессиональной деятельности;
			\item способность понимать роль науки в развитии цивилизации, соотношение науки и техники, иметь представление о связанных с ними современных социальных и этических проблемах, понимать ценность научной рациональности и ее исторических типов.
		\end{itemize}

\newpage
\section{Введение}
	Развитие инфраструктуры современных мегаполисов невозможно без многоэтажных жилых, производственных и офисных зданий,
		в которых эксплуатируются несколько лифтов, объединенных в группы. В этой связи актуальным видится вопрос
		совершенствования взаимной работы таких групп с целью повышения их эффективности.
		Это позволит как уменьшить число лифтов группы, так и сократить энергетические затраты существующих,
		что особенно уместно на фоне постоянного роста тарифов на электроэнергию.

Автоматизированные системы управления активно приходят в повседневную жизнь человечества. Сначала, это были системы для управления производственным процессом на крупных предприятиях, теперь данные системы решают и бытовые задачи. Одной из таких задач является доставка человека с одного этажа на другой. Данная задача достаточно подробно описана в книге С.В. Васильева [1], там имеются абстрактная модель, логическая модель и логический вывод.

С предложенным логическим выводом справилась бы система автоматического доказательства теорем А.А. Ларионова [2] или программа на языке Prolog, в которой было бы возможным реализовать представленные позитивно образованные формулы. Таким образом возникает задача реализовать процесс построения логического вывода на языке Prolog.
